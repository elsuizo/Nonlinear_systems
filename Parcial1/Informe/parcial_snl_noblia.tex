\documentclass[10pt]{article}
%-*-*-*-*-*-*-*-*-*-*-*-*-*-*-*-*-*-*-*Packages--*--*--*--*--*--*--*--*--*--*--*--*--*
\usepackage{amsmath}
\usepackage{tikz}
\usepackage{epigraph}
\usepackage{lipsum}
\usepackage{mdframed}
\usepackage{fancyhdr}
 \usepackage[utf8]{inputenc}
\usepackage{hyperref} 
\usepackage[spanish]{babel}
%\usepackage{simpsons}
%\usepackage[dvips]{graphicx}
\usepackage{amsmath}
\usepackage{amsthm}
%\usepackage{textcomp}
\usepackage{amssymb}
\usepackage{latexsym}
\usepackage{graphicx}
%\usepackage[ansinew]{inputenc}
\usepackage{color}
%\usepackage{pstricks, pst-node, pst-plot, pst-circ}
%\usepackage{moredefs}
%\usepackage{pstricks}
%\usepackage{pst-circ}
%\usepackege{pst-node}
%\usepackage{pst-plot}
\usepackage{moredefs}
%\usepackage{mcode}
%\usepackage[above]{placeins}
\usepackage{fancybox}
\usepackage{subfig}
\usepackage{float}
%\usepackage{mcode} para colorear codigo de matlab
\usepackage{xcolor}
\usepackage{wallpaper}
\usepackage{textcomp}
%-*-*-*-*-*-*-*-*-*-*-*-*-*-*-*-*-*-*--*--*--*--*--*--*--*--*--*--*--*--*--*--*

%*****************************definiciones*******************************
\oddsidemargin -0.2in \textwidth 7.0in \topmargin -0.9in \textheight
9.0in
\parindent 3em
\hyphenation{e-jem-plo
e-rro-res de-pen-dien-te co-rrien-te res-pues-ta fi-gu-ran
cons-tan-tes o-pe-ra-cion i-lu-mi-na-cion}

\pagestyle{fancy}
\headheight 50pt 
\renewcommand{\arrayrulewidth}{3pt} 
%@@@@@@@@@@@@@@@@@@@@@@@@@@@cabecera y pie de pagina @@@@@@@@@@@@@@@@@@@@@@@@@@@@@@@@@@@@@@@@@@@@@@@@@@@@@@

\rhead{\includegraphics[width=.075\textwidth]{logo_iaci2.eps}}
\chead{}
\lhead{Sistemas nolineales}
\rfoot{1º cuatrimestre 2013  }
\cfoot{\thepage}
\lfoot{\bf{Universidad Nacional de Quilmes}}
%@@@@@@@@@@@@@@@@@@@@@@@@@@@@@@@@@@@@@@@@@@@@@@@@@@@@@@@@@@@@@@@@@@@@@@@@@@@@@@@@@@@@@@@@@@@@@@@@@@@@@@@@@@@

%::::::::::::::::::::::Colores varios:::::::::::::::::::::::::::::::::::::::::::::::::::::::::::::::::::::::

\definecolor{azul-claro}{rgb}{0.335,0.89,1}
\definecolor{gainsboro}{rgb}{0.86,0.86,0.86}
\definecolor{mediumseagreen}{rgb}{0.24,0.7,0.44}
\definecolor{moccasin}{rgb}{1,0.89,0.71}
\definecolor{cornflowerblue}{rgb}{0.39,0.58,0.93}
\definecolor{lightgray}{rgb}{0.83,0.83,0.83}
\definecolor{darkgrey}{rgb}{0.66,0.66,0.66}
\definecolor{darkslategray}{rgb}{0.18,0.31,0.31}
\definecolor{lavender}{rgb}{0.9,0.9,0.98}
\definecolor{azure}{rgb}{0.94,1,1}
\definecolor{honeydew}{rgb}{0.94,1,0.94}

%:::::::::::::::::::::::::::::::::::::::::::::::::::::::::::::::::::::::::::::::::::::::::::::::::::::::::::::
%Requiere \usepackage{xcolor}
\renewcommand{\arrayrulewidth}{1pt} 
\newcommand{\mcaja}[1]{%
{{\fboxsep 10pt \fboxrule 1pt%
\fcolorbox{black}{orange}{%
\color{black} \Huge #1}}}
}
\newcommand{\nuevobox}[1]{%
{{\fboxsep 14pt \fboxrule 1pt%
\fcolorbox{black}{darkgrey}{%
\color{lavender} \huge #1}}}
}
\newcommand{\ssection}[1]{\section[#1]{\mcaja{#1}}}
\makeatletter
\newcommand{\sect}[1]{\subsection[#1]{\nuevobox{#1}}}
\makeatletter
\def\section{\@ifstar\unnumberedsection\numberedsection}
\def\numberedsection{\@ifnextchar[%]
\numberedsectionwithtwoarguments\numberedsectionwithoneargument}
\def\unnumberedsection{\@ifnextchar[%]
\unnumberedsectionwithtwoarguments\unnumberedsectionwithoneargument}
\def\numberedsectionwithoneargument#1{\numberedsectionwithtwoarguments[#1]{#1}}
\def\unnumberedsectionwithoneargument#1{\unnumberedsectionwithtwoarguments[#1]{#1}}
\def\numberedsectionwithtwoarguments[#1]#2{%
\ifhmode\par\fi
\removelastskip
\vskip 3ex\goodbreak
\refstepcounter{section}%
\begingroup
%\noindent
\leavevmode\large\bfseries\raggedright\mcaja%%
\thesection\ #2\par\nobreak
\endgroup
\noindent\hrulefill\nobreak
\vskip 2ex\nobreak
\addcontentsline{toc}{section}{%
\protect\numberline{\thesection}%
#1}%
}
\def\unnumberedsectionwithtwoarguments[#1]#2{%
\ifhmode\par\fi
\removelastskip
\vskip 3ex\goodbreak
% \refstepcounter{section}%
\begingroup
\noindent
\leavevmode\Large\bfseries\raggedright
% \thesection\
#2\par\nobreak
\endgroup
\noindent\hrulefill\nobreak
\vskip 0ex\nobreak
\addcontentsline{toc}{section}{%
%
\protect\numberline{\thesection}%
#1}%
}
\makeatother
%%%Cap\’itulos
\usepackage{helvet}
%\usepackage{psboxit,pstcol}
\makeatletter
\def\@makechapterhead#1{%
{\parindent \z@ \raggedright \reset@font
\hbox to \hsize{%
\rlap{\raisebox{-2.5em}{\raisebox{\depth}{%%% Necesita la imagen "imgCapitulo"
\includegraphics[width=10em]{logo_iaci2.eps}}}}%
\rlap{\hbox to 6em{\hss
\reset@font\sffamily\fontsize{8em}{8em}\selectfont\black
\thechapter\hss}}%
\hspace{10em}%
\vbox{%
\advance\hsize by -10em
\reset@font\fontfamily{hv}\bfseries\Large
#1
\par
}%
}}%
\vskip 5pt
\hrulefill
\vskip 50pt
}
\makeatother
%----------------------------------------------------------------------------------------
% Theorems
%\newtheorem{lem}{Lemma}[section]
%\newtheorem{coro}{Corolario}[section]
%\newtheorem{teo}{Torema}[section]
%\newtheorem{defi}{Definición}[section]

\newmdtheoremenv{lem}{Lema}[section]
\newmdtheoremenv{coro}{Corolario}[section]
\newmdtheoremenv{teo}{Teorema}[section]
\newmdtheoremenv{defi}{Definición}[section]
%----------------------------------------------------------------------------------------
\definecolor{titlepagecolor}{cmyk}{0.73,.73,0,.37}
\newcommand\titlepagedecoration{%
\begin{tikzpicture}[remember picture,overlay,shorten >= -10pt]

\coordinate (aux1) at ([yshift=-15pt]current page.north east);
\coordinate (aux2) at ([yshift=-410pt]current page.north east);
\coordinate (aux3) at ([xshift=-4.5cm]current page.north east);
\coordinate (aux4) at ([yshift=-150pt]current page.north east);

\begin{scope}[titlepagecolor!40,line width=12pt,rounded corners=12pt]
\draw
  (aux1) -- coordinate (a)
  ++(225:5) --
  ++(-45:5.1) coordinate (b);
\draw[shorten <= -10pt]
  (aux3) --
  (a) --
  (aux1);
\draw[opacity=0.3,titlepagecolor,shorten <= -10pt]
  (b) --
  ++(225:2.2) --
  ++(-45:2.2);
\end{scope}
\draw[titlepagecolor,line width=8pt,rounded corners=8pt,shorten <= -10pt]
  (aux4) --
  ++(225:0.8) --
  ++(-45:0.8);
\begin{scope}[titlepagecolor!70,line width=6pt,rounded corners=8pt]
\draw[shorten <= -10pt]
  (aux2) --
  ++(225:3) coordinate[pos=0.45] (c) --
  ++(-45:3.1);
\draw
  (aux2) --
  (c) --
  ++(135:2.5) --
  ++(45:2.5) --
  ++(-45:2.5) coordinate[pos=0.3] (d);   
\draw 
  (d) -- +(45:1);
\end{scope}
\end{tikzpicture}%
}


%----------------------------------------------------------------------------------------
%	TITLE PAGE
%----------------------------------------------------------------------------------------
\newcommand*{\plogo}{\fbox{$\mathcal{PL}$}}
\newcommand*{\titleGP}{\begingroup % Create the command for including the title page in the document
\centering % Center all text
\vspace*{\baselineskip} % White space at the top of the page

\rule{\textwidth}{1.6pt}\vspace*{-\baselineskip}\vspace*{2pt} % Thick horizontal line
\rule{\textwidth}{0.4pt}\\[\baselineskip] % Thin horizontal line

{\LARGE Sistemas No-Lineales  \\[0.3\baselineskip] Primer parcial}\\[0.2\baselineskip] % Title

\rule{\textwidth}{0.4pt}\vspace*{-\baselineskip}\vspace{3.2pt} % Thin horizontal line
\rule{\textwidth}{1.6pt}\\[\baselineskip] % Thick horizontal line

\scshape % Small caps
 Martín Noblía 
\vspace*{2\baselineskip} % Whitespace between location/year and editors

Profesoras: \\[\baselineskip]
{\Large Virginia Mazzone \\ Mariana Suarez \par}
\vfill % Whitespace between editor names and publisher logo_iaci2
\includegraphics[width=.175\textwidth]{logo_iaci2.eps}

% Editor list
{\itshape Universidad Nacional de Quilmes \par} % Editor affiliation

\vfill % Whitespace between editor names and publisher logo

%\plogo \\[0.3\baselineskip] % Publisher logo
{\scshape 2015} \\[0.3\baselineskip] % Year published
%{\large THE PUBLISHER}\par % Publisher

\endgroup}

%----------------------------------------------------------------------------------------
%	BLANK DOCUMENT
%----------------------------------------------------------------------------------------


\begin{document}
\pagestyle{empty} 

\titleGP
\titlepagedecoration
\newpage
\pagestyle{fancy}

\section{Problema 1}
Dado el sistema no lineal:

\begin{equation}
     \label{eq:problema1}
      \left\{
	       \begin{array}{ll}
               \dot{x}=x-y-(x^2 + \frac{3}{2}y^2)x\\[5pt] %truco para agregar espacio
		\dot{y}=x+y-(x^2 + \frac{1}{2}y^2)y

	       \end{array}
	     \right.
\end{equation}

Analizar la existencia de ciclos límites.

\subsection{Resolución}
Podemos expresar el sistema \eqref{eq:problema1}:
\begin{equation}
\mathbf{x}=
\begin{bmatrix}
 x(t) \\
 y(t)
 \end{bmatrix}
 \end{equation}

 \begin{equation}
\mathbf{f}=
\begin{bmatrix}
 f_{1}(x(t),y(t)) \\
 f_{2}(x(t),y(t))
 \end{bmatrix}
 \end{equation}
 Donde:


\begin{equation}
     \label{eq:problema1:efes}
      \left\{
	       \begin{array}{ll}
               f_{1}=x-y-(x^2 + \frac{3}{2}y^2)x\\[5pt] %truco para agregar espacio
               f_{2}=x+y-(x^2 + \frac{1}{2}y^2)y

	       \end{array}
	     \right.
\end{equation}


Asi nuestro sistema dinámico no lineal \eqref{eq:problema1} escrito en forma compacta queda:

\begin{equation}
 \dot{\mathbf{x}}=\mathbf{f}(\mathbf{x}) \label{eq:forma_vec}
\end{equation}
Donde $\mathbf{x}$ es el vector de estados del sistema.
Para investigar la presencia de órbitas periódicas vamos a necesitar primero de la teoría(definiciones y teoremas) en la cual nos basaremos y mediante la cual quedará más clara la resolución.

\begin{defi}[Puntos de equilibrios (PE)]
 Un punto $\mathbf{x}=\mathbf{x}^{*}$ en el espacio de estados es un punto de equilibrio (PE) de \eqref{eq:forma_vec} si tiene la propiedad de 
que cuando el estado inicial del sistema es $\mathbf{x}^{*}$, el estado permanece en $\mathbf{x}^{*}$ en todo tiempo futuro.
\end{defi}

Ya que $\mathbf{x}^{*}=cte$ entonces los puntos de equilibrio de \eqref{eq:forma_vec} son:

\begin{equation}
 \mathbf{f}(\mathbf{x})=\mathbf{0} \label{eq:pe}
\end{equation}

Además sabemos que podemos inferir el comportamiento cualitativo en las cercanias del/los punto/s de equilibrio/s, bajo ciertas condiciones 
que se enuncian en el siguiente teorema:

\begin{teo}[Hartman-Grobman]

Consideremos el sistema no lineal planar $ \dot{\mathbf{x}}=\mathbf{f}(\mathbf{x})$, con $\mathbf{f}$ lo suficientemente suave. Supongamos que $\mathbf{x}^{*}$
es un punto de equilibrio del sistema y que $A=\dfrac{\partial \mathbf{f}}{\partial \mathbf{x}}\bigg\vert_{\mathbf{x}=\mathbf{x}^{*}}$ no tiene autovalores
nulos o imaginarios puros. Entonces existe un difeomorfismo $\mathbf{h}$ definido en un entorno del equilibrio $\mathbf{U}$ de $\mathbf{x}^{*}$ ($\mathbf{h}:\mathbf{U} \longrightarrow \mathbb{R}^{2}$) de modo tal que "lleva"
las trayectorias del sistema no lineal sobre las del sistema linealizado. En particular $\mathbf{h}(\mathbf{x}^{*})=\mathbf{0}$
\label{hart}

\end{teo}

\begin{defi}[Ciclos límite]
Un ciclo límite es una órbita cerrada y aislada. Un sistema oscila cuando tiene una solución periódica no trivial:

\begin{equation}
\mathbf{x}(t+T)=\mathbf{x}(t) \; ; \hspace{.5cm} t\geq 0 \hspace{.5cm} \text{para algún} \quad  T > 0
\end{equation}
La imagen de una solución periódica en el retrato de fase es la de una órbita periódica u órbita cerrada. 
\end{defi}

Recordemos que la matriz $A$ del teorema ~\ref{hart} es la matriz Jacobiana evaluada en el PE y se define:
\begin{equation}
A=\dfrac{\partial \mathbf{f}}{\partial \mathbf{x}}\bigg\vert_{\mathbf{x}=\mathbf{x}^{*}}=
\begin{bmatrix}
 \dfrac{\partial f_1}{\partial x_1}& \dfrac{\partial f_1}{\partial x_2} \\ 
\dfrac{\partial f_2}{\partial x_1}& \dfrac{\partial f_2}{\partial x_2}
\end{bmatrix} 
\bigg\vert_{\mathbf{x}=\mathbf{x}^{*}}
\label{eq:jaco}
\end{equation}

\begin{lem}[Poincaré-Bendixon\footnote{Ivar Otto Bendixson (August 1, 1861 – 1935) was a Swedish mathematician.}]
Consideremos el sistema \eqref{eq:forma_vec} y sea $\mathbf{M}$ un conjunto  cerrado y acotado de $\mathbb{R}^2$ tal que:
\begin{itemize}
\item $\mathbf{M}$ no contiene puntos de equilibrio, o contiene solo un punto de equilibrio($\mathbf{p}$) tal que la matriz Jacobiana $A=\dfrac{\partial \mathbf{f}}{\partial \mathbf{x}}\bigg\vert_{\mathbf{x}=\mathbf{p}}$ tiene autovalores con parte real positiva(por lo tanto, el punto de equilibrio es un foco inestable o un nodo inestable)

\item Toda trayectoria que comienza en $\mathbf{M}$ permanece en $\mathbf{M}$ para todo tiempo futuro
\end{itemize}
Entonces $\mathbf{M}$ contiene una órbita periódica de \eqref{eq:forma_vec}
\end{lem}


\begin{lem}[Criterio de Bendixon]

Si sobre una región simplemente conexa $\mathbf{D} \subset \mathbb{R}^2$ la expresión $\nabla \; \mathbf{f}=\dfrac{\partial f_1}{\partial x_1}+\dfrac{\partial f_2}{\partial x_2}$ es no identicamente cero y no cambia de signo, entonces el sistema \eqref{eq:forma_vec} no posee órbitas periódicas situadas enteramente en $\mathbf{D}$

\label{bendixon}

\end{lem}

Necesitamos entonces calcular los puntos de equilibrio del sistema, lo hacemos a traves de la fórmula \eqref{eq:pe}, o sea que nos queda por resolver el siguiente sistema de ecuaciones no lineales:

\begin{equation}
     \label{eq:problema1:pes}
      \left\{
	       \begin{array}{ll}
               x-y-(x^2 + \frac{3}{2}y^2)x=0\\[5pt] %truco para agregar espacio
               x+y-(x^2 + \frac{1}{2}y^2)y=0

	       \end{array}
	     \right.
\end{equation}



\newpage

\begin{thebibliography}{6}
 \bibitem{Principal}{ Nonlinear Systems, third edition. Hassan K. Khalil,ISBN 0-13-067389-7 }
 \bibitem{alternativo}{Applied Nonlinear Control. Jean-Jacques E. Slotine, Weiping Li, ISBN 0-13-040890-5}
\end{thebibliography}




\end{document}
