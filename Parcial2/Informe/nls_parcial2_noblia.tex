\documentclass[10pt]{article}
%-*-*-*-*-*-*-*-*-*-*-*-*-*-*-*-*-*-*-*Packages--*--*--*--*--*--*--*--*--*--*--*--*--*

\usepackage[usenames,dvipsnames]{pstricks}
\usepackage{epsfig}
\usepackage{pst-grad} % For gradients
\usepackage{pst-plot} % For axes

\usepackage{amsmath}
\usepackage{tikz}
\usepackage{epigraph}
\usepackage{lipsum}
\usepackage{mdframed}
\usepackage{fancyhdr}
 \usepackage[utf8]{inputenc}
\usepackage{hyperref} 
\usepackage[spanish]{babel}
%\usepackage{simpsons}
%\usepackage[dvips]{graphicx}
\usepackage{amsmath}
\usepackage{amsthm}
%\usepackage{textcomp}
\usepackage{amssymb}
\usepackage{latexsym}
\usepackage{graphicx}
%\usepackage[ansinew]{inputenc}
\usepackage{color}
%\usepackage{pstricks, pst-node, pst-plot, pst-circ}
%\usepackage{moredefs}
%\usepackage{pstricks}
%\usepackage{pst-circ}
%\usepackege{pst-node}
%\usepackage{pst-plot}
\usepackage{moredefs}
%\usepackage{mcode}
%\usepackage[above]{placeins}
\usepackage{fancybox}
\usepackage{subfig}
\usepackage{float}
%\usepackage{mcode} para colorear codigo de matlab
\usepackage{xcolor}
\usepackage{wallpaper}
\usepackage{textcomp}
\usepackage{circuitikz}
\oddsidemargin -0.2in \textwidth 7.0in \topmargin -0.9in \textheight
9.0in
\parindent 3em
\hyphenation{e-jem-plo
e-rro-res de-pen-dien-te co-rrien-te res-pues-ta fi-gu-ran
cons-tan-tes o-pe-ra-cion i-lu-mi-na-cion}

\pagestyle{fancy}
\headheight 50pt 
\renewcommand{\arrayrulewidth}{3pt} 
%-------------------------------------------------------------------------
% Cabecera y pie de pagina
%-------------------------------------------------------------------------

\rhead{\includegraphics[width=.075\textwidth]{logo_iaci2.eps}}
\chead{}
\lhead{Sistemas No-Lineales}
\rfoot{1º cuatrimestre 2015  }
\cfoot{\thepage}
\lfoot{\bf{Universidad Nacional de Quilmes}}
%-------------------------------------------------------------------------
% colores varios 
%-------------------------------------------------------------------------

\definecolor{azul-claro}{rgb}{0.335,0.89,1}
\definecolor{gainsboro}{rgb}{0.86,0.86,0.86}
\definecolor{mediumseagreen}{rgb}{0.24,0.7,0.44}
\definecolor{moccasin}{rgb}{1,0.89,0.71}
\definecolor{cornflowerblue}{rgb}{0.39,0.58,0.93}
\definecolor{lightgray}{rgb}{0.83,0.83,0.83}
\definecolor{darkgrey}{rgb}{0.66,0.66,0.66}
\definecolor{darkslategray}{rgb}{0.18,0.31,0.31}
\definecolor{lavender}{rgb}{0.9,0.9,0.98}
\definecolor{azure}{rgb}{0.94,1,1}
\definecolor{honeydew}{rgb}{0.94,1,0.94}

%:::::::::::::::::::::::::::::::::::::::::::::::::::::::::::::::::::::::::::::::::::::::::::::::::::::::::::::
%Requiere \usepackage{xcolor}
\renewcommand{\arrayrulewidth}{1pt} 
\newcommand{\mcaja}[1]{%
{{\fboxsep 10pt \fboxrule 1pt%
\fcolorbox{black}{orange}{%
\color{black} \Huge #1}}}
}
\newcommand{\nuevobox}[1]{%
{{\fboxsep 14pt \fboxrule 1pt%
\fcolorbox{black}{darkgrey}{%
\color{lavender} \huge #1}}}
}
\newcommand{\ssection}[1]{\section[#1]{\mcaja{#1}}}
\makeatletter
\newcommand{\sect}[1]{\subsection[#1]{\nuevobox{#1}}}
\makeatletter
\def\section{\@ifstar\unnumberedsection\numberedsection}
\def\numberedsection{\@ifnextchar[%]
\numberedsectionwithtwoarguments\numberedsectionwithoneargument}
\def\unnumberedsection{\@ifnextchar[%]
\unnumberedsectionwithtwoarguments\unnumberedsectionwithoneargument}
\def\numberedsectionwithoneargument#1{\numberedsectionwithtwoarguments[#1]{#1}}
\def\unnumberedsectionwithoneargument#1{\unnumberedsectionwithtwoarguments[#1]{#1}}
\def\numberedsectionwithtwoarguments[#1]#2{%
\ifhmode\par\fi
\removelastskip
\vskip 3ex\goodbreak
\refstepcounter{section}%
\begingroup
%\noindent
\leavevmode\large\bfseries\raggedright\mcaja%%
\thesection\ #2\par\nobreak
\endgroup
\noindent\hrulefill\nobreak
\vskip 2ex\nobreak
\addcontentsline{toc}{section}{%
\protect\numberline{\thesection}%
#1}%
}
\def\unnumberedsectionwithtwoarguments[#1]#2{%
\ifhmode\par\fi
\removelastskip
\vskip 3ex\goodbreak
% \refstepcounter{section}%
\begingroup
\noindent
\leavevmode\Large\bfseries\raggedright
% \thesection\
#2\par\nobreak
\endgroup
\noindent\hrulefill\nobreak
\vskip 0ex\nobreak
\addcontentsline{toc}{section}{%
%
\protect\numberline{\thesection}%
#1}%
}
\makeatother
%%%Cap\’itulos
\usepackage{helvet}
%\usepackage{psboxit,pstcol}
\makeatletter
\def\@makechapterhead#1{%
{\parindent \z@ \raggedright \reset@font
\hbox to \hsize{%
\rlap{\raisebox{-2.5em}{\raisebox{\depth}{%%% Necesita la imagen "imgCapitulo"
\includegraphics[width=10em]{logo_iaci2.eps}}}}%
\rlap{\hbox to 6em{\hss
\reset@font\sffamily\fontsize{8em}{8em}\selectfont\black
\thechapter\hss}}%
\hspace{10em}%
\vbox{%
\advance\hsize by -10em
\reset@font\fontfamily{hv}\bfseries\Large
#1
\par
}%
}}%
\vskip 5pt
\hrulefill
\vskip 50pt
}
\makeatother
%----------------------------------------------------------------------------------------
% Theorems
%\newtheorem{lem}{Lemma}[section]
%\newtheorem{coro}{Corolario}[section]
%\newtheorem{teo}{Torema}[section]
%\newtheorem{defi}{Definición}[section]

\newmdtheoremenv{lem}{Lema}[section]
\newmdtheoremenv{coro}{Corolario}[section]
\newmdtheoremenv{teo}{Teorema}[section]
\newmdtheoremenv{defi}{Definición}[section]
%----------------------------------------------------------------------------------------
\definecolor{titlepagecolor}{cmyk}{0.73,.73,0,.37}
\newcommand\titlepagedecoration{%
\begin{tikzpicture}[remember picture,overlay,shorten >= -10pt]

\coordinate (aux1) at ([yshift=-15pt]current page.north east);
\coordinate (aux2) at ([yshift=-410pt]current page.north east);
\coordinate (aux3) at ([xshift=-4.5cm]current page.north east);
\coordinate (aux4) at ([yshift=-150pt]current page.north east);

\begin{scope}[titlepagecolor!40,line width=12pt,rounded corners=12pt]
\draw
  (aux1) -- coordinate (a)
  ++(225:5) --
  ++(-45:5.1) coordinate (b);
\draw[shorten <= -10pt]
  (aux3) --
  (a) --
  (aux1);
\draw[opacity=0.3,titlepagecolor,shorten <= -10pt]
  (b) --
  ++(225:2.2) --
  ++(-45:2.2);
\end{scope}
\draw[titlepagecolor,line width=8pt,rounded corners=8pt,shorten <= -10pt]
  (aux4) --
  ++(225:0.8) --
  ++(-45:0.8);
\begin{scope}[titlepagecolor!70,line width=6pt,rounded corners=8pt]
\draw[shorten <= -10pt]
  (aux2) --
  ++(225:3) coordinate[pos=0.45] (c) --
  ++(-45:3.1);
\draw
  (aux2) --
  (c) --
  ++(135:2.5) --
  ++(45:2.5) --
  ++(-45:2.5) coordinate[pos=0.3] (d);   
\draw 
  (d) -- +(45:1);
\end{scope}
\end{tikzpicture}%
}


%----------------------------------------------------------------------------------------
%	TITLE PAGE
%----------------------------------------------------------------------------------------
\newcommand*{\plogo}{\fbox{$\mathcal{PL}$}}
\newcommand*{\titleGP}{\begingroup % Create the command for including the title page in the document
\centering % Center all text
\vspace*{\baselineskip} % White space at the top of the page

\rule{\textwidth}{1.6pt}\vspace*{-\baselineskip}\vspace*{2pt} % Thick horizontal line
\rule{\textwidth}{0.4pt}\\[\baselineskip] % Thin horizontal line

{\LARGE Sistemas No-Lineales  \\[0.3\baselineskip] Segundo parcial}\\[0.2\baselineskip] % Title

\rule{\textwidth}{0.4pt}\vspace*{-\baselineskip}\vspace{3.2pt} % Thin horizontal line
\rule{\textwidth}{1.6pt}\\[\baselineskip] % Thick horizontal line

\scshape % Small caps
 Martín Noblía 
\vspace*{2\baselineskip} % Whitespace between location/year and editors

Profesoras: \\[\baselineskip]
{\Large Virginia Mazzone \\ Mariana Suarez \par}
\vfill % Whitespace between editor names and publisher logo_iaci2
\includegraphics[width=.175\textwidth]{logo_iaci2.eps}

% Editor list
{\itshape Universidad Nacional de Quilmes \par} % Editor affiliation

\vfill % Whitespace between editor names and publisher logo

%\plogo \\[0.3\baselineskip] % Publisher logo
{\scshape 2015} \\[0.3\baselineskip] % Year published
%{\large THE PUBLISHER}\par % Publisher

\endgroup}

%----------------------------------------------------------------------------------------
%	BLANK DOCUMENT
%----------------------------------------------------------------------------------------


\begin{document}
\pagestyle{empty} 

\titleGP
\titlepagedecoration
\newpage
\pagestyle{fancy}

%-------------------------------------------------------------------------
% Problema 1
%-------------------------------------------------------------------------

\section{Problema 1}
Utilizando linealización exacta por realimentación, diseñar un control que logre
seguimiento asintótico de referencias constantes para el siguiente sistema:

\begin{equation}
\dot{\mathbf{x}}=
\begin{bmatrix}
 x_{1}x_{2} - x_{1}^{3}\\
 x_{1} \\
 -x_{3} \\
 x_{1}^{2} + x_{2}
\end{bmatrix}
+ 
\begin{bmatrix}
0 \\
2 + 2x_{3} \\
1 \\
0
\end{bmatrix}
u
\end{equation}

\begin{equation}
y = x_{4}
\end{equation}

\subsection{Resolución}

Como primera observación, el sistema posee la forma afin al control, osea esta 
enmarcado en la siguiente estructura:

\begin{align}
\begin{array}{rcl}
    \dot{\mathbf{x}}& = &\mathbf{f}(\mathbf{x}) + \mathbf{g}(\mathbf{x}) u \\
    y & = & h(\mathbf{x})\end{array}
\label{ec:afin_al_control}
\end{align}

Donde:
\begin{equation}
    \mathbf{f(\mathbf{x})}=
\begin{bmatrix}
 x_{1}x_{2} - x_{1}^{3}\\
 x_{1} \\
 -x_{3} \\
 x_{1}^{2} + x_{2}
\end{bmatrix}
\label{ec:f}
\end{equation}

\begin{equation}
    \mathbf{g(\mathbf{x})}=
\begin{bmatrix}
 0\\
 2+ 2x_{3} \\
 1\\
 0
\end{bmatrix}
\label{ec:g}
\end{equation}

\begin{equation}
    h(\mathbf{x}) = x_{4}
\end{equation}

Una de las definiciones más importantes de la teoría de linealización exacta por
realimentación es la de grado relativo de un sistema no-lineal, la misma nos dice
cuales son los alcances de nuestra linealización y si se puede realizar.

\begin{defi}[Grado Relativo]
    El sistema \eqref{ec:afin_al_control} Donde: $\mathbf{f}$, $\mathbf{g}$, $h$, definidas
    en un dominio $D\in \mathbb{R}^{n}$, son suficientemente suaves, tienen grado relativo $r$
    con $1\leq r \leq n$ en una región $D_{0}\in D$ si

\begin{equation}
    \frac{\partial \psi_{i}}{\partial \mathbf{x}}\mathbf{g}(\matbf{x}) = 0 \hspace{5em} i=1,2,\dots , r-1
\end{equation}

\begin{equation}
    \frac{\partial \psi_{r}}{\partial \mathbf{x}} \mathbf{g}(\matbf{x}) \neq 0  
\end{equation}
       Donde:

\begin{equation}
    \psi_{1}(\mathbf{x}) = h(\mathbf{x}) 
     \label{ec:restriccion_phi_1}
\end{equation}
y
\begin{equation}
     \psi_{i+1}(\mathbf{x}) = \frac{\partial \psi_{i}}{\partial \mathbf{x}}\mathbf{f}(\matbf{x}) 
     \label{ec:restriccion_phi_2}
\end{equation}

\end{defi}   


Si el sistema tiene grado relativo ($GR$) $r$, entonces es linealizable entrada-salida. Si
tiene $GR=n$ entonces es linealizable tanto estrada-salida como entrada-estado.

La manera practica de verificar el $GR$ es calcular las derivadas sucesivas de la salida, hasta
que aparezca explicitamente la entrada $u$, en nuetro caso:

\begin{align}
\begin{array}{rcl}
    \dot{y}&=&\dot{x_{4}}= x_{1}^{2} + x_{2} \\
    \ddot{y} &=& 2x_{1}\dot{x_{1}} + \dot{x_{2}} \\
    &= & 2x_{1} (x_{1}x_{2}-x_{1}^3) + x_{1} + (2 + 2x_{3})u
\end{array}
\end{align}

Ya que la entrada $u$ apareció en la segunda derivada de la salida, decimos que el $GR$ del sistema \eqref{ec:afin_al_control} es $2$.
Ello significa que solo es linealizable entrada-salida, y que solo dos estados de los cuatro seran \textit{vistos} por la ley de control.
El grado relativo esta definido en
\begin{equation}
    D_{0} = \left\{\matbf{x} \in \mathbb{R}^{4} | x_{3} \neq -1 \right\}
\end{equation}

Para lograr dicha linealización es necesario primero realizar un cambio de variables,
%
\begin{align}
    \mathbf{z} = T(\mathbf{x}) = \begin{bmatrix} \phi(\mathbf{x}) \\ \psi(\mathbf{x}) \end{bmatrix} \triangleq \begin{bmatrix} \eta \\ \xi \end{bmatrix}
\label{ec:difeo}
\end{align}
%
para que las variables de estado pasen a ser $\eta$, que es inobservable desde $y$ y se elige de modo que no aparezca la entrada $u$ en su ecuación y $\xi$, que representa la transferencia entrada-salida de un integrador. Representado de esta manera, se dice que el sistema se encuentra en su forma normal
%
\begin{align}
\begin{array}{rcl}
\dot{\eta} & = & f_0(\eta,\xi) \\
\dot{\xi} & = & A_c \xi + B_c \beta(\mathbf{x})^{-1}[u-\alpha(\mathbf{x})] \\
y & = & C_c \xi
\end{array}
\label{ec:sistnor}
\end{align}
%
donde $\xi$ y $\eta$ $\in \R$, ($A_c$, $B_c$, $C_c$) es la forma canónica de dos integradores, es decir 


\begin{equation}
A_{c}=
\begin{bmatrix}
  0 & 1\\
  0 & 0
\end{bmatrix}
\end{equation}

\begin{equation}
B_{c}=
\begin{bmatrix}
  0\\ 
  1 
\end{bmatrix}
\end{equation}

\begin{equation}
C_{c}=
\begin{bmatrix}
  0 \, 1 
\end{bmatrix}
\end{equation}

Donde:

%
\begin{align}
    f_{0}(\eta,\xi) = \dfrac{\partial \phi}{\partial \mathbf{x}}f(\mathbf{x})\vert_{\mathbf{x}=T^{-1}(\mathbf{z})}
\label{ec:fsubcero}
\end{align}
%
%
\begin{align}
\begin{array}{rcl}
    \beta(\mathbf{x}) = \dfrac{1}{(\partial \psi_{2} / \partial \mathbf{x} ) g(\mathbf{x}) } & \textnormal{y} & \alpha(\mathbf{x}) = - \dfrac{(\partial \psi_{2} / \partial \mathbf{x}) f(\mathbf{x})}{(\partial \psi_{2} / \partial \mathbf{x}) g(\mathbf{x})}
\end{array}
\label{alfaybeta}
\end{align}

En nuestro caso el esquema que tenemos es el siguiente:
\begin{align}
    \mathbf{z} = T(\mathbf{x}) = \begin{bmatrix} \phi_{1}(\mathbf{x}) \\ \phi_{2}(\mathbf{x}) \\ \psi_{1}(\mathbf{x})  \\ \psi_{2}(\mathbf{x})\end{bmatrix} \triangleq \begin{bmatrix}\mathbf{\phi} \\ \mathbf{\psi} \end{bmatrix} \triangleq \begin{bmatrix} \eta \\ \xi \end{bmatrix}
\label{ec:difeo_aumentado}
\end{align}
%
Procedemos tomando como $\psi_{1}(\mathbf{x})=h(\mathbf{x})=x_{4}$, luego utilizando \eqref{ec:restriccion_phi_2} calculamos $\psi_{2}(\mathbf{x})=x_{1}^{2}+x_{2}$. Prosegimos calculando las funciones $\alpha(\mathbf{x})$ y $\beta(\mathbf{x})$, lo que resulta en:

\begin{equation}
    \beta(\mathbf{x})=\frac{1}{2+2x_{3}}
\end{equation}

\begin{equation}
    \alpha(\mathbf{x}) = -\frac{2x_{1}^{2}x_{2}-2x_{1}^{4}+x_{1}}{2+2x_{3}}
\end{equation}

Antes de continuar con el calculo del difeomorfismo, es conveniente asegurar que el sistema es de minima fase
ya que si posee esa propiedad se puede asegurar que puede ser estabilizado al punto de equilibrio($\mathbf{x}=\mathbf{0}$). 
Ya que la forma normal \eqref{ec:sistnor} tiene una parte \textit{externa} representada por las variables $\xi$, y una parte \textit{interna}
representada por las variables $\eta$. El control $u=\alpha(\mathbf{x})+\beta(\mathbf{x})v$ linealiza la parte \textit{externa} y hace inobservable la parte \textit{interna}. Haciendo $\xi=0$ en la primera ecuación \eqref{ec:sistnor} tenemos la contraparte nolineal de lo que se conoce en el contexto de Control lineal clásico como dinámica de los ceros.

\begin{equation}
    \dot{\eta} = \mathbf{f}_{0}(\eta, 0)
    \label{ec:f_0}
\end{equation}
que es la contraparte nolineal de $\dot{\eta}=A_{c}\eta$. El sistema nolineal se dice de mínima fase si la dinámica de los ceros tiene un PEAE
en el dominio de interes. La ecuación \eqref{ec:f_0}  representa la dinámica de los ceros en las nuevas variables, podemos sin embargo caracterizarla en las variables originales, notando que:

%
\begin{align}
\begin{array}{c}
y(t) \equiv 0 \Rightarrow \xi(t) \equiv 0 \Rightarrow u(t) \equiv \alpha(x(t))
\end{array}
\end{align}
Entonces mantener la salida identicamente cero implica que la solución de la ecuación de estados debe quedar confinada al conjunto:

\begin{align}
\begin{array}{c}
    Z^{*} = \lbrace x \in D_0 \hspace{0.1cm} \vert \hspace{0.1cm} \psi_{1}(\mathbf{x}) = \psi_{2}(\mathbf{x}) = \cdots = 0 \rbrace
\end{array}
\end{align}
%
y la entrada debe ser: $u=u^{*}(\mathbf{x}):= \alpha(\mathbf{x})|_{\mathbf{x}\in Z^{*}}$, con ello la dinámica de los ceros es la dinámica restringida:

\begin{equation}
    \dot{\mathbf{x}} = \mathbf{f}^{*}(\mathbf{x}) = [\mathbf{f}(\mathbf{x})+ \mathbf{g}(\mathbf{x})\alpha(\mathbf{x})]|_{\mathbf{x}\in Z^{*}}
\end{equation}

Entonces en nuestro caso: $\psi_{1}(\mathbf{x})=0$ $\rightarrowtail$ $x_{4}=0$ y $\psi_{2}(\mathbf{x})=0$ $\rightarrowtail$ $x_{2} = -x_{1}^{2}$
Y la dinámica restringida en $Z^{*}$ queda:


\begin{align}
    \mathbf{f}^{*} = \begin{bmatrix} -2x_{1}^{3} \\ 4x_{1}^{4} \\ -x_{3}-\frac{-4x_{1}^{4}+x_{1}}{2+2x_{3}} \\ 0\end{bmatrix}
    \label{ec:dinamica_ceros}
\end{align}
%-------------------------------------------------------------------------
% falta ver si esta bien esto o hay que hacerlo de nuevo
%-------------------------------------------------------------------------


\section{Problema 2}
En la figura, se muestra la configuración \textit{Boost}(elevador) de un comvertidor electrónico dc-dc

\begin{itemize}
        \item Obtenga el modelo en espacio de estados considerando como $u$ a la posición de la llave. Note que cuando la llave esta en la posición $1$ hay dos circuitos en funcionamiento de forma independiente.
        \item Considerando la superficie $S_{v}=v-V=0$, estabilice utilizando regimen deslizante.
        \item Idem al anterior pero considerando la superficie $S_{I}=i-I=0$. Asignando a las constantes el siguiente valor numerico:
            	\begin{center}
	                R=1 \hspace{1cm} L=1 \hspace{1cm} C=1 \hspace{1cm} E=1 \hspace{1cm} V=1
	            \end{center}
                Simule el lazo de control. Obtenga el retrato de fase donde pueda verse el regime deslizante. Analice si se verifica el objetivo de contol 
        \item Utilizando la salida $y=v\frac{1}{2}Cv^{2}+\frac{1}{2}Li^{2}$(energía almacenada), linealice el sistema utilizando modo deslizantes. Simule utilizando los mismos parametros que en el item anterior.
        \end{itemize}





\begin{figure}[H]
\centering    
\begin{circuitikz}[american voltages]
\draw
  (0,0) to [short, *-] (10,0)
  to [R, l_=$R$] (10,4) 
  (0,0) to [battery, l=$E$] (0,4) 
  to [L, l=$L$] (2,4)
  to [cspst, l=s] (3,4) 
  (4,0) to [short,*-o] (4,3)
  (10,4) to [short,*-] (7,4)
  (7,4) to [short,*-o] (5,4)
  (7,4) to [C, l_=$C$] (7,0) 
  (5,4) node[below] {$u=0$}
  (4,3) node[right] {$u=1$};
  \end{circuitikz}
  \caption{Convertidor Boost}
  \end{figure}

\subsection{Resolución}
Comenzamos planteando las ecuaciones del sistema para ambas posiciones de la llave\\
\textbf{Posición de u: $u=1$:}\\

\[
-E+v_{L}=0 \Rightarrow E=v_{L}=L\dfrac{di_{L}}{dt}
\]

\[
v_{R}+v_{c}=i_{c} R + v_{c} = 0 \Rightarrow v_{c} = - i R  = - \dfrac{dv_{c}}{dt} CR
\]  

\begin{equation}
\left\lbrace  \begin{array}{rcl} \label{Boost_u=1}
		\dfrac{di}{dt}	&=	&\dfrac{E}{L}\\
						&	&\\
		\dfrac{dv_{c}}{dt} &=	&-\dfrac{1}{cR} v_{c}
\end{array}
\right. 
\end{equation}

\textbf{Posición de u: $u=0$:}\\

\[
E=v_{L}+v_{c}=L\dfrac{di}{dt} + v_{c} \Rightarrow \dfrac{di}{dt}=\dfrac{E}{L}-\dfrac{1}{L} v_{c}
\]

\[
i_{L}=i_{c}+i_{R} \Rightarrow i_{R}=i_{L}-i_{c} = i_{L}- \dfrac{dv_{c}}{dt}c
\]

\[
-v_{c}+v_{R}=0 \Rightarrow v_{c}=v_{R}=i_{R} R= i_{L} R - cR \dfrac{dv_{c}}{dt} \Rightarrow \dfrac{dv_{c}}{dt} = \dfrac{1}{C}i - \dfrac{1}{CR}v_{c}
\]  

\begin{equation}	\label{boost_u=0}
\left\lbrace  \begin{array}{rcl}
		\dfrac{di}{dt}	&=	&\dfrac{E}{L}-\dfrac{1}{L}v_{c}\\
						&	&\\
		\dfrac{dv_{c}}{dt} &=	&\dfrac{1}{c}i-\dfrac{1}{cR}v_{c}\\
\end{array}
\right. 
\end{equation}

Utilizando los sistemas \eqref{Boost_u=1} y \eqref{boost_u=0} y la acción de $u$, ampliamos el sistema para escribirlo en una forma mas compacta

 \begin{equation}  \label{Sistema_boost}
\left\lbrace  \begin{array}{rcl}
		\dfrac{di}{dt}	&=	&\dfrac{E}{L}+\dfrac{1}{L}v_{c} (u-1)\\
						&	&\\
		\dfrac{dv_{c}}{dt} &=	&-\dfrac{1}{c} i\; u + \left( -\dfrac{1}{c}i + \dfrac{1}{cR}v_{c} \right) \left(u-1\right) \\
\end{array}
\right. 
\end{equation}
Reescribiendo \eqref{Sistema_boost} y usando $i_{L}=x_{1}$, $v_{c}=x_{2}$ obtenemos

\begin{equation}
\left\lbrace \begin{array}{lcl} \label{sistema 1}
			 \dot{x_{1}}	&=	&\dfrac{E}{L}-\dfrac{1}{L}x_{2} + \dfrac{1}{L}x_{2}\,u\\
			 				&	&\\
			 \dot{x_{2}}	&=	&\dfrac{1}{c}x_{1}-\dfrac{1}{cR}x_{2}-\dfrac{1}{c}x_{1}\,u
\end{array}
\right. 
\end{equation}

Finalmente, es importante remarcar que el sistema \eqref{sistema 1}, muestra en su estructura básica, un esquema a fin al control. Es esto ultimo, una de las condiciones esenciales en el planteo de las estrategias de control a desarrollar, es decir, es condición necesaria para que estos planteos puedan ser formulados. En este sentido, vemos en la ecuación \eqref{afin al control} el esquema general de un sistema en su forma afin:

\begin{equation} \label{afin al control}
\dot{x}=f(x)+g(x)u
\end{equation}

En donde $u$ representa la señal de entrada al sistema. Analizando ahora el sistema de estudio propuesto, las funciones vectoriales $f(x)$ y $g(x)$ vienen dadas por:\\

\begin{multicols}{2}
\[
f(x)=\left[ \begin{array}{c}
			-\dfrac{1}{L}x_{2}\\
			\\
			\dfrac{1}{c}x_{1}-\dfrac{1}{cR}x_{2}
\end{array}
\right]
\]

\[
g(x)=\left[ \begin{array}{c}
			\dfrac{1}{L}x_{2}\\
			\\
			-\dfrac{1}{c}x_{1}\\
\end{array}
\right]
\]
\end{multicols}





%
\end{document}
